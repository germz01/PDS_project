\chapter{Parallel architecture design} % (fold)
\label{cha:parallel_architecture_design}
    \pagenumbering{arabic}
    In this first chapter I'll describe the details regarding the \textbf{parallel architecture design} of
    the Image \textit{watermarking} project. In order to follow a certain criterion during the design phase,
    I've followed Figure $6.2$ of \cite{DSPM}, and, in general, the
    concepts of chapter $6$. In Figure \ref{fig:parallel_architecture} you can see a representation of the
    parallel architecture that will be described in the following sections.

    \section{Task definition} % (fold)
    \label{sec:task_definition}
        One of the most important choices to be taken during the design of a parallel application consist in
        defining what has to be considered as the smallest unit of work for the processes involved in the
        computation, that is, the definition of what a \textbf{task} is. For the Image watermarking project,
        I've chosen to consider each image that has to be processed as a distinct task. Actually, each image is
        coupled with some additional informations, in order to ease the workers in the processing phase. More
        details about this are provided in chapter \ref{cha:implementation_structure_and_implementation_details}.
    % section task_definition (end)

    \section{Finding concurrency design space} % (fold)
    \label{sec:finding_concurrency_design_space}
        Following the order given by Figure $6.2$ of \cite{DSPM}, and keeping in mind what a task is in the
        context of the watermarking project, the next step, in the \textit{finding concurrency} design phase,
        consist in selecting the \textbf{decomposition} and \textbf{dependency analysis} pattern to apply in the
        parallel application. For the former, I've chosen the \textit{task decomposition} pattern, remaining
        coherent with the task definition, and for the latter I've chosen the \textit{data sharing} pattern.
        The task decomposition pattern models the division of a complex application in a set of tasks, which in
        this case are represented by the images, while the data sharing pattern models the accesses to a shared
        data structure, which in this case is represented by a queue containing the tasks, which mimics a
        \textbf{stream of images}. The processes retrieves one single task every time they access the queue.
        Again, more details on the implementation will be provided in chapter
        \ref{cha:implementation_structure_and_implementation_details}.
    % section finding_concurrency_design_space (end)

    \section{Algorithm structure design space} % (fold)
    \label{sec:algorithm_structure_design_space}
        In order to figure out a parallel program structure closer to an actual parallel program, what is been
        decided in sections \ref{sec:task_definition} and \ref{sec:finding_concurrency_design_space} is refined
        in the \textit{algorithm structure} design space. I've decided to organize the parallel algorithm
        \textit{by task}, and, in particular, using the \textit{task parallelism} pattern, since, as it is
        imaginable by reading the past sections, every task is \textbf{independent} by itself. The following
        scheduling is used:
        \begin{enumerate}
            \item A (master) process is in charge of loading the images taken from a directory which is given in
            input by the user. Every image is then pushed into the stream queue, with a certain delay, also
            given in input by the user. The entire procedure is builded in order to simulate a stream channel
            that outputs images.
            \item A (worker) process, concurrently with the other (worker) processes, pops out a task from the stream queue.
            \item The process apply the watermark on the image retrieved from the queue.
            \item Finally, the process saves the image, with the watermark applied on it, in an output
            directory, which is given in input by the user.
        \end{enumerate}
        It is important to keep in mind that steps 1 and 2 (and, obviously steps 3 and 4) are performed
        concurrently. A process that is waiting for the queue to be available will be awoken as soon as a new
        image is loaded into the stream, as will be described in chapter
        \ref{cha:implementation_structure_and_implementation_details}. The scheduling described in this section
        is represented in Figure \ref{fig:parallel_architecture}.
    % section algorithm_structure_design_space (end)

    \section{Supporting structures design space} % (fold)
    \label{sec:supporting_structures_design_space}
        In the \textit{supporting structures design space} the structures, that is, the
        patterns, suitable to support the implementation of the algorithms planned in section
        \ref{sec:algorithm_structure_design_space} are investigated. Among the two available choices,
        the \textit{data structures} and the \textit{program structures}, I've chosen the latter, and, in
        particular, the well known \textbf{Master/Worker} pattern, which models the concurrent execution of a
        \textit{bag of tasks} on a collection of identical workers. In the context of the watermarking project,
        the master is represented by the main process, which loads the images from the image directory,
        composes an object coupling the loaded image with other informations and
        pushes the resulting task into the stream queue, and the workers are represented by a set of
        threads that all execute the same function, which accesses the stream queue and, after retrieving a
        \textit{task}, apply the watermark (given in input by the user) and finally saves the modified image
        into the output directory.
    % section supporting_structures_design_space (end)

    \section{Implementation mechanisms design space} % (fold)
    \label{sec:implementation_mechanisms_design_space}
        Finally, I describe the design space which models the mechanisms needed to support the
        parallel computing abstractions typical of parallel programming, that is, the \textit{implementation
        mechanisms} design space. In order to guarantee a safe access to the stream queue, I've chosen to put a
        mutex on it. The worker that try to access the queue while it is in use by another worker will by
        blocked, and will wait till the queue become available again. This kind of choice will influence the
        final performances by introducing some overhead, as explained in \cite{DSPM} and \cite{spp}.
        This is the only shared data structure that is present in the program; all the other activities take
        place concurrently without interfering with eachother. As explained in the past sections, since the task
        are independent, in this particular case the worker don't need to communicate, the only kind of
        "synchronizations" consist in checking if the stream queue is empty or not.

        \begin{figure}[t]
        \centering
            \begin{tikzpicture}
                \begin{scope}[every node/.style={circle,thick,draw}]
                    \node (A) at (-1,0) {master};
                    \node (C) at (6,1.5) {worker};
                    \node (D) at (6,-1.5) {worker};
                \end{scope}

                \node[rectangle,thick,draw] (F) at (-4,0) {\rotatebox{90}{watermark}};
                \node[rectangle,thick,draw] (G) at (9,0) {\rotatebox{90}{output directory}};
                \node[rectangle,thick,draw] (H) at (-3, 3) {image directory};
                \node[rectangle,thick,draw] (I) at (2, 0) {stream};

                \begin{scope}[>={Stealth[black]},every node/.style={fill=white,circle},
                every edge/.style={draw=black,very thick}]
                    \path [->] (A) edge node {\footnotesize retrieves} (F);
                    \path [->] (C) edge[bend left=30] node {\footnotesize saves} (G);
                    \path [->] (D) edge[bend right=30] node {\footnotesize saves} (G);
                    \path [->] (A) edge node {\footnotesize retrieves} (H);
                    \path [->] (A) edge node {\footnotesize inserts} (I);
                    \path [->] (C) edge node {\footnotesize retrieves} (I);
                    \path [->] (D) edge node {\footnotesize retrieves} (I);
                \end{scope}

                \path (C) -- (D) node [black, font=\LARGE, midway, sloped] {$\dots$};
        \end{tikzpicture}
        \caption{Representation of the parallel architecture described in this chapter.}
        \label{fig:parallel_architecture}
        \end{figure}

    % section implementation_mechanisms_design_space (end)

% chapter parallel_architecture_design (end)
