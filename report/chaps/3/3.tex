\chapter{Implementation structure and implementation details} % (fold)
\label{cha:implementation_structure_and_implementation_details}
    In this chapter I'll give some more technical details about the choices I've taken during the implementation
    of the watermarking project.
    \section{Directory tree} % (fold)
    \label{sec:directory_tree}
        The structure I've chosen to follow for my project is quite trivial, and it is represented in Figure
        \ref{fig:directory_tree}. You can find the complete project on the
        \href{https://github.com/germz01/PDS_project}{\textit{GitHub}} page.
    % section directory_tree (end)
    \section{Task implementation} % (fold)
    \label{sec:task_implementation}
        As stated in chapter \ref{cha:parallel_architecture_design}, each task is composed by an image and some
        additional information. On the implementation side I've decided to represent a task by using a
        \texttt{struct image\_t} containing a pointer to the image to be processed and the image's name.
        Inserting the image's name simplify the comparison between processed and unprocessed images.
    % section task_implementation (end)
    \section{Simulating the stream of images} % (fold)
    \label{sec:simulating_the_stream_of_images}
        In order to simulating a stream of images, I've chosen to use a \texttt{std::deque}. While the emitter,
        that is, the master thread, inserts the tasks into the front of the queue, the workers extracts the tasks
        from the back of the queue, in a FIFO way. The \texttt{delay} parameter, given in input, is used to
        simulate the time passing between two insertions (in microseconds). Being the queue a shared resource,
        in order to insert/extract a task, it is mandatory to take the \texttt{lock} on the queue, which in my
        implementation it is represented by a \texttt{std::lock\_guard} for the emitter, and by a
        \texttt{std::unique\_lock} for the workers. A \texttt{std::condition\_variable} ensures that, if some
        workers try to access the queue while another worker is extracting a task, they'll be notified as soon as
        the queue become available again. Once that there are no more images to process, the emitter inserts a
        \texttt{nullptr} into the queue.
    \section{Applying the watermark} % (fold)
    \label{sec:applying_the_watermark}
        From the implementation point of view, every image is a $X{\times}Y$ matrix, being $X$ the image's wight
        and $Y$ the image's height. Every worker has a \texttt{std::vector<point\_t>} containing the coordinates
        of every black point composing the watermark image, which is used during the watermark's application.
        For every point contained in the vector, a black pixel is generated in the original image, and this
        realizes the watermark's application.
    % section applying_the_watermark (end)
    \section{Loading and saving an image} % (fold)
    \label{sec:loading_and_saving_an_image}
        The only functionalities I've used from the CImg library \cite{cimg} are the two functions for loading
        and saving an image. As I'll describe in chapter \ref{cha:experimental_validation}, this functions can
        be considered as the bottlenecks for both the standard and fastflow program's version. As you can see
        from the sample of an output in appendix \ref{cha:how_to_run_the_program}, the functions for loading
        and saving an image are the ones that take the bigger part of the computation.
    % section loading_and_saving_an_image (end)
    \begin{figure}[b!]
        \dirtree{%
            .1 PDS\_project.
            .2 imgs \DTcomment{contains the sample images that were used during the
            project's development}.
            .2 report \DTcomment{contains this report}.
            .2 results \DTcomment{contains the csv files with the experimental results gathered during the
            tests}.
            .2 sources \DTcomment{contains the project's code base}.
            .3 scripts.
            .4 extract\_results.py.
            .4 plot\_results.py.
            .3 sources\_ff.
            .4 fastflow.
            .4 main.cpp \DTcomment{fastflow version of the program}.
            .4 main.
            .4 makefile.
            .3 main.cpp \DTcomment{standard version of the program}.
            .3 main.
            .3 utilities.hpp.
            .3 makefile.
        }
        \caption{Directory tree for the watermarking project.}
        \label{fig:directory_tree}
    \end{figure}
    % section simulating_the_stream_of_images (end)
% chapter implementation_structure_and_implementation_details (end)
