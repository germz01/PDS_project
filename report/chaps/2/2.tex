\chapter{Performance modeling} % (fold)
\label{cha:performance_modeling}
    In this chapter I'll describe the choices I made to model the \textbf{performance} of the code implementing
    the watermarking project, following the concepts and using the terminology of \cite{DSPM}, chapter
    $5$, and \cite{spp}, chapter $2$. I'll also spend some words about the path I followed to collect the
    data for the models' verification.
    \section{Parameters' selection} % (fold)
    \label{sec:parameters}
        First things first, for defining the building blocks of the performance model for the image
        watermarking project, I've chosen the \textit{parameters} to be considered in order to shape the
        performance of the program. Having the goal of describing the program's performance in
        the most human-friendly way, I've chosen to consider the basic performance measures of interest in
        parallel/distributed computing, that is, the \textbf{completion time} ($T_C$), the \textbf{speedup}
        ($s(n)$), the \textbf{scalability} ($\mathit{scalab}(n)$) and finally the \textbf{efficiency}
        ($\epsilon(n)$). For a detailed explanation of this parameters, see \cite{DSPM} and \cite{spp}.
    % section parameters (end)

    \section{Collecting the data} % (fold)
    \label{sec:collecting_the_data}
        There are essentially two version of the program implementing the watermarking project: a
        \textbf{standard one}, which includes the sequential and parallel implementation implemented using only
        the functionalities provided by the C++ standard library, and a \textbf{fastflow one}, that includes only
        the parallel implementation which uses \cite{fastflow}. Each one of the parallel
        versions was tested using a parallelism degree that ranges from $1$ to $10$. Given the parameters to be
        observed, the data to conduct an analytical valuation of the program's performances was collected at run
        time, using the utilities provided by the standard C++ library \textit{chrono}. The act of collecting the
        data interferes with the overall computation by adding some overhead, that will influence the final
        performances. I've used a Python script to compute the measures of section \ref{sec:parameters} both for
        the standard and the fastflow version. The results were then stored in four diffente csv files, two for
        the execution of the two versions of the program on my laptop, a MacBook Pro, and two for the execution
        on the department's machine. The results can bee founded in the \textbf{results} directory, on the
        project's \href{https://github.com/germz01/PDS_project}{\textit{GitHub page}}.
    % section collecting_the_data (end)

    \section{Formalising the model} % (fold)
    \label{sec:formalising_the_model}
        In order to ease the description of the models describing the sequential and parallel version of the
        program, I'll use some symbols:
        \begin{description}
            \item[$T_l$] represents the time taken to load an image from the image directory using the utilities
            provided by \cite{cimg}.
            \item[$T_p$] represents the time taken to process, that is, to apply the watermark on an image.
            \item[$T_s$] represents the time taken to save an image in the output directory using the utilities
            provided by \cite{cimg}.
            \item[$d$] represents the delay introduced between every image, in order to simulate the behavior of
            a stream of images.
            \item[$n_i$] represents the number of images processed during at the end of the execution. For the
            sake of the simulation we image that this number is unknown, but I want to introduce it anyway in the
            formulas to give a better understanding of the performances.
        \end{description}
        \subsection{Sequential version} % (fold)
        \label{sub:sequential_version}
            As we could imagine, the sequential version of the program is the simpler one to model. In order to
            represent the time taken by an image to be processed we can use the following formula:

            \begin{equation}
                T_l + T_p + T_s + d.
                \label{eq:sequential}
            \end{equation}

            Knowing equation \ref{eq:sequential}, we can represent the \textbf{completion time} of the sequential
            version of the program with the formula:

            \begin{equation}
                T_C = n_i * \left ( T_l + T_p + T_s + d \right ).
                \label{eq:completion_time_sequential}
            \end{equation}

        % subsection sequential_version (end)
        \subsection{Parallel version} % (fold)
        \label{sub:parallel_version}
            For the parallel version of the program we have a slightly different situation. We can imagine the
            master/worker pattern like a two-stage pipeline, in which the first stage is used to load the images
            into the stream queue, and the second stage processes the images by first applying the watermark and
            then saving the modified image into the output directory. The \textbf{completion time} can be
            represented by the following formula:

            \begin{equation}
                T_C = \mathit{max} \left \{ \frac{T_p + T_s}{n_w}, T_l + d \right \},
                \label{eq:completion_time_parallel}
            \end{equation}

            where $n_w$ represents the number of workers, that is, the number of threads that are concurrently
            processing the images loaded by the master thread, and, as a mat{}ter of fact, corresponds to the
            parallelism degree given in input by the user. We can consider $T_p + T_s$ as the \textbf{latency}
            for the parallel version of the program, that is, the time spent between the moment a given
            activity receives input data and the moment the same activity delivers the output data
            corresponding to the input. It has to be kept in mind that equation
            \ref{eq:completion_time_parallel} abstracts from the overheads introduced by the
            mechanisms described in sections \ref{sec:implementation_mechanisms_design_space} and
            \ref{sec:collecting_the_data}.
        % subsection parallel_version (end)
    % section formalising_the_model (end)

% chapter performance_modeling (end)
