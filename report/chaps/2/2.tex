\chapter{Performance modeling} % (fold)
\label{cha:performance_modeling}

    \section{Selection of the parameters} % (fold)
    \label{sec:parameters}
        First things first, for defining the building blocks of the performance model for the image
        watermarking project, I've chosen the \textit{parameters} to be considered in order to model the
        performance of the program. Having the goal of describing the performance of the parallel application in
        the most human-friendly way, I've chosen the basic performance measures of interest in
        parallel/distributed computing, that is, the \textbf{latency} ($L$), the \textbf{completion time}
        ($T_C$), the \textbf{service time} ($T_S$), the \textbf{bandwidth} ($\mathcal{B}$), the \textbf{speedup}
        ($s(n)$), the \textbf{scalability} ($\mathit{scalab}(n)$) and finally the \textbf{efficiency}
        ($\epsilon(n)$). A detailed explanation of this measures can be founded in \cite{DSPM}, chapter 5.
    % section parameters (end)

    \section{Collecting the data} % (fold)
    \label{sec:model_s_usage}
        Given the parameters to be observed, the data to conduct an analytical valutation of the program's
        performances was collected at run time, using the utilities provided by the standard C++ library
        \textit{chrono}. For the experimental results, go to chapter \ref{cha:experimental_validation}.
    % section model_s_usage (end)

% chapter performance_modeling (end)
