\chapter{How to run the program} % (fold)
\label{cha:how_to_run_the_program}
    \pagenumbering{Roman}
    \setcounter{page}{2}
    As stated in chapter \ref{cha:performance_modeling}, there are two versions of the program implementing
    the watermarking project. Both versions can be runned via the main executable produced by the compilation
    via \textit{make}, or using the Python script for the performances' measurement. The command for the
    first kind of execution is the following:

    \begin{minted}[bgcolor=LightGray,fontsize=\scriptsize]{bash}
        ./main <images-directory> <watermark-file> <parallelism-degree> <output-directory> <delay>
    \end{minted}

    The second kind of execution is a bit more complex, but the script itself provides all the informations
    in order to safely run the simulation.

    \begin{minted}[bgcolor=LightGray,fontsize=\scriptsize]{bash}
        usage: extract_results.py [-h] [-e {standard,fastflow}] [-i IMAGEDIR]
                      [-w WATERMARK] [-o OUTPUTDIR] [-l LOOP] [-d DELAY]
                      [-n NAME]

        This script collects data about the execution time of the main program.

        optional arguments:
          -h, --help            show this help message and exit
          -e {standard,fastflow}, --executable {standard,fastflow}
                                Which executable should be tested.
          -i IMAGEDIR, --imagedir IMAGEDIR
                                The path leading to the directory containing the
                                images.
          -w WATERMARK, --watermark WATERMARK
                                The path leading to the watermark file.
          -o OUTPUTDIR, --outputdir OUTPUTDIR
                                The path leading to the output directory
          -l LOOP, --loop LOOP  Number of iterations the main program have to be
                                executed.
          -d DELAY, --delay DELAY
                                The delay represents the time that the emitter have to
                                wait before sending a new image into the stream.
          -n NAME, --name NAME  Name of the file in which the results will be saved.
    \end{minted}
% chapter how_to_run_the_program (end)
