\chapter{Various} % (fold)
\label{cha:various}
    \section{How to compile the program} % (fold)
    \label{sec:how_to_compile_the_program}
        For compiling both the versions implementing the watermarking project it is sufficient to run the
        \texttt{make} command in the \texttt{source} directory if you want to compile the version using only the
        standard C$++$ library functionalities, or in the \texttt{source\_ff} directory if you want to compile
        the version using fastflow.
    % section how_to_compile_the_program (end)
    \section{How to run the program} % (fold)
    \label{sec:how_to_run_the_program}
        As stated in chapter \ref{cha:performance_modeling}, there are two versions of the program implementing
        the watermarking project. Both versions can be runned via the \texttt{main} executable produced by the
        compilation via \textit{make}, or using the Python script for the performances' measurement.
        \subsection{Standard execution} % (fold)
        \label{sub:standard_execution}
            For the standard execution, navigate to the \texttt{source} directory if you want to launch the
            program implemented with only the standard C$++$ library functionalities, or navigate to the
            \texttt{source\_ff} directory if you want to launch the program implemented using fastflow.
            For both the executions, the command is the following:

            \begin{minted}[bgcolor=LightGray,fontsize=\scriptsize]{bash}
                ./main <images-directory> <watermark-file> <parallelism-degree> <output-directory> <delay>
            \end{minted}

            Where:
            \begin{description}
                \item[\texttt{<image-directory>}] is the path leading to the directory containing the images to
                be processed;
                \item[\texttt{<watermark-file>}] is the path leading to the watermark image;
                \item[\texttt{<parallelism-degree>}] is the number of workers to use in the parallel version of
                the standard implementation, or in the fastflow one;
                \item[\texttt{<output-directory>}] is the path leading to the directory in which the images
                have to be saved;
                \item[\texttt{<delay>}] represents the amout of time, in $\mu s$, between two insertions in the
                stream queue, as described in chapter
                \ref{cha:implementation_structure_and_implementation_details};
            \end{description}
            Here is an example for the standard C++ library execution:

            \begin{minted}[bgcolor=LightGray,fontsize=\scriptsize]{bash}
                ./main ../imgs/ ../watermark.jpg 1 ../output_dir/ 100
            \end{minted}

            And here is one for the fastflow execution:

            \begin{minted}[bgcolor=LightGray,fontsize=\scriptsize]{bash}
                ./main ../../imgs/ ../../watermark.jpg 1 ../../output_dir/ 100
            \end{minted}
        % subsection standard_execution (end)
        \subsection{Python execution} % (fold)
        \label{sub:python_execution}
            The Python execution is the one I've used in order to collect the data that you can see displayed in
            Figure \ref{fig:performances}. This kind of execution is a bit more complex, but the script itself
            provides all the informations in order to safely run the simulation.

            \begin{minted}[bgcolor=LightGray,fontsize=\tiny]{bash}
                usage: extract_results.py [-h] [-e {standard,fastflow}] [-i IMAGEDIR]
                              [-w WATERMARK] [-o OUTPUTDIR] [-l LOOP] [-d DELAY]
                              [-n NAME]

                This script collects data about the execution time of the main program.

                optional arguments:
                  -h, --help            show this help message and exit
                  -e {standard,fastflow}, --executable {standard,fastflow}
                                        Which executable should be tested.
                  -i IMAGEDIR, --imagedir IMAGEDIR
                                        The path leading to the directory containing the
                                        images.
                  -w WATERMARK, --watermark WATERMARK
                                        The path leading to the watermark file.
                  -o OUTPUTDIR, --outputdir OUTPUTDIR
                                        The path leading to the output directory
                  -l LOOP, --loop LOOP  Number of iterations the main program have to be
                                        executed.
                  -d DELAY, --delay DELAY
                                        The delay represents the time that the emitter have to
                                        wait before sending a new image into the stream.
                  -n NAME, --name NAME  Name of the file in which the results will be saved.
            \end{minted}

            The main difference between this execution and the one described in section
            \ref{sub:standard_execution} is that the Python execution runs the selected version of the program
            \texttt{LOOP} times, and, for every iteration, collects the data about the parameters describes in
            chapter \ref{cha:performance_modeling}. Finally the collected data are saved in a csv file named
            after \texttt{NAME}.
        % subsection python_execution (end)
        \subsection{Sequential execution} % (fold)
        \label{sub:sequential_execution}
            If you want to execute the standard version's sequential implementation, you have to use the
            command described in \ref{sub:standard_execution} and insert $0$ as
            \texttt{<parallelism-degree>}, while being in the directory containing the standard version's
            compiled file (the project's directory tree is represented in Figure \ref{fig:directory_tree}).
        % subsection sequential_execution (end)
    % section how_to_run_the_program (end)
    \section{Output example} % (fold)
    \label{sec:output_example}
        Here is a typical output sequence obtained by using $1$ as parallelism degree. Both the two program's
        versions uses this formatting style to output the times taken by the various part of the computation.

        \begin{align*}
            &\text{PARALLELISM DEGREE:} \ 1 \\
            &\text{COMPLETION TIME:} \ 4701698 \ \mu s \\
            &\text{MEAN PROCESSING TIME:} \ 255.512 \ \mu s \\
            &\text{MEAN LOADING TIME:} \ 103834 \ \mu s \\
            &\text{MEAN SAVING TIME:} \ 137938 \ \mu s \\
            &\text{MEAN CREATION TIME:} \ 21.6667 \ \mu s \\
            &\text{PROCESSED IMAGES:} \ 43
        \end{align*}

        If the kind of execution that has been chosen is the Python one, everything that is printed into the
        consolle's log is parsed and collected into the csv files, which can be seen on the project's
        \href{https://github.com/germz01/PDS_project}{\textit{GitHub}} page.
    % section output_example (end)
% chapter various (end)
