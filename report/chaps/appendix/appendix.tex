\chapter{How to run the program} % (fold)
\label{cha:how_to_run_the_program}
    As stated in chapter \ref{cha:performance_modeling}, there are two versions of the program implementing
    the watermarking project. Both versions can be runned via the main executable produced by the compilation
    via \textit{make}, or using the Python script for the performances' measurement. The command for the
    first kind of execution is the following:

    \begin{minted}[bgcolor=LightGray,fontsize=\scriptsize]{bash}
        ./main <images-directory> <watermark-file> <parallelism-degree> <output-directory> <delay>
    \end{minted}

    Where \texttt{<images-directory>} is the path leading to the directory containing the images to be processed,
    \texttt{<watermark-file>} is the path leading to the watermark image, \texttt{<parallelism-degree>} is the
    number of workers to use in the parallel version of the standard implementation, or in the fastflow one,
    \texttt{<output-directory>} is the path leading to the directory in which the images have to be saved and
    finally \texttt{delay} represents the amout of time, in microseconds, between the two insertions in the stream
    queue, as described in chapter \ref{cha:implementation_structure_and_implementation_details}.
    If you want to execute the standard version's sequential implementation, you have to insert $0$ as
    parallelism degree while being in the directory containing the standard version's compiled file (you can
    find the project's directory tree in Figure \ref{fig:directory_tree}). \\
    The second kind of execution is a bit more complex, but the script itself provides all the informations
    in order to safely run the simulation.

    \begin{minted}[bgcolor=LightGray,fontsize=\scriptsize]{bash}
        usage: extract_results.py [-h] [-e {standard,fastflow}] [-i IMAGEDIR]
                      [-w WATERMARK] [-o OUTPUTDIR] [-l LOOP] [-d DELAY]
                      [-n NAME]

        This script collects data about the execution time of the main program.

        optional arguments:
          -h, --help            show this help message and exit
          -e {standard,fastflow}, --executable {standard,fastflow}
                                Which executable should be tested.
          -i IMAGEDIR, --imagedir IMAGEDIR
                                The path leading to the directory containing the
                                images.
          -w WATERMARK, --watermark WATERMARK
                                The path leading to the watermark file.
          -o OUTPUTDIR, --outputdir OUTPUTDIR
                                The path leading to the output directory
          -l LOOP, --loop LOOP  Number of iterations the main program have to be
                                executed.
          -d DELAY, --delay DELAY
                                The delay represents the time that the emitter have to
                                wait before sending a new image into the stream.
          -n NAME, --name NAME  Name of the file in which the results will be saved.
    \end{minted}

    Here is a typical output provided by the execution of the standard/fastflow program's version.

    \begin{minted}[bgcolor=LightGray,fontsize=\scriptsize]{bash}
        PARALLELISM DEGREE: 0
        COMPLETION TIME: 7560410 μs
        MEAN LATENCY: 234.884 μs
        MEAN LOADING TIME: 72277.7 μs
        MEAN SAVING TIME: 101028 μs
        MEAN CREATION TIME: nan μs
        PROCESSED IMAGES: 43
    \end{minted}
% chapter how_to_run_the_program (end)
