\chapter{Implementation structure and implementation details} % (fold)
\label{cha:implementation_structure_and_implementation_details}
    In this chapter I'll give some more technical details about the choices I've taken during the implementation
    phase of the watermarking project. For an in depth analysis of the implementation details see the code on the
    project's \href{https://github.com/germz01/PDS_project}{\textit{GitHub}} page, or \cite{cpp}.
    \section{Directory tree} % (fold)
    \label{sec:directory_tree}
        The structure I've chosen to follow for my project is quite trivial, and it is represented in Figure
        \ref{fig:directory_tree}. See the \href{https://github.com/germz01/PDS_project}{\textit{GitHub}} page
        for the complete project.
    % section directory_tree (end)
    \section{Task implementation} % (fold)
    \label{sec:task_implementation}
        As stated in chapter \ref{cha:parallel_architecture_design}, each task is composed by an image and some
        additional information. On the implementation side I've decided to represent a task by using a
        \texttt{struct image\_t} containing a pointer to the image to be processed and the image's name.
        Inserting the image's name simplify the comparison between processed and unprocessed images, and assures
        the consistency between the names of the original image and the processed one.
    % section task_implementation (end)
    \section{Simulating the stream of images} % (fold)
    \label{sec:simulating_the_stream_of_images}
        For simulating a stream of images, I've chosen to use a \texttt{std::deque}. While the emitter,
        that is, the master thread, inserts the tasks into the front of the queue, the workers extracts the tasks
        from the back of the queue, in a FIFO way. The \texttt{delay} parameter, given in input, is used to
        simulate the time passing between two insertions (in $\mu s$). Being the queue a shared resource,
        in order to insert/extract a task, it is mandatory to take the \texttt{lock} on the queue, which in my
        implementation it is represented by a \texttt{std::lock\_guard} for the emitter, and by a
        \texttt{std::unique\_lock} for the workers. A \texttt{std::condition\_variable} ensures that, if some
        workers try to access the queue while another worker is extracting a task, they'll be notified as soon as
        the queue become available again. Once that there are no more images to process, the emitter inserts a
        \texttt{nullptr} into the queue.
    \section{Master and workers} % (fold)
    \label{sec:master_and_workers}
        In the standard version of the program the master thread is represented by the \texttt{main} function,
        which execute the function \texttt{fill\_queue} till there are no more images to upload. In the fastflow
        version of the program this role is taken by a fastflow's \texttt{Emitter}, which basically performs
        the same function as the standard version's master thread.
        The number of workers to generate during every execution is given by the user via the
        \texttt{<parallelism-degree>} parameter, as described in appendix \ref{sec:how_to_run_the_program}. At
        run-time, every worker is represented by a thread. In the standard version of the program, the workers
        are contained inside an array of type \text{std::thread}, while in the fastflow version of the program
        it is used a \texttt{std::vector<ff\_node *>}, and then a \texttt{ff\_farm<>}, following the examples
        provided in \cite{DSPM}.
    % section master_and_workers (end)
    \section{Applying the watermark} % (fold)
    \label{sec:applying_the_watermark}
        From the implementation point of view, every image is a $X{\times}Y$ matrix, being $X$ the image's wight
        and $Y$ the image's height. Every worker has a \texttt{std::vector<point\_t>} containing the coordinates
        of every black pixel composing the watermark image, which is used during the watermark's application.
        For every entry in the vector, a black pixel is generated in the original image, and this
        realizes the watermark's application.
    % section applying_the_watermark (end)
    \section{Loading and saving an image} % (fold)
    \label{sec:loading_and_saving_an_image}
        The only functionalities I've taken from \cite{cimg} are the two functions for loading
        and saving an image. As I'll describe in chapter \ref{cha:experimental_validation}, this functions can
        be considered as the bottlenecks for both the standard and fastflow program's version. As you can see
        from the sample of an output in appendix \ref{sec:how_to_run_the_program}, the functions for loading
        and saving an image are the ones that take the bigger part of the computation.
    % section loading_and_saving_an_image (end)
    \begin{figure}[b!]
        \dirtree{%
            .1 PDS\_project.
            .2 imgs \DTcomment{contains the sample images that were used during the
            project's development}.
            .2 report \DTcomment{contains this report}.
            .2 results \DTcomment{contains the csv files with the experimental results gathered during the
            tests}.
            .2 sources \DTcomment{contains the project's code base}.
            .3 scripts.
            .4 extract\_results.py.
            .4 plot\_results.py.
            .3 sources\_ff.
            .4 fastflow-master.
            .4 main.cpp \DTcomment{fastflow version of the program}.
            .4 main.
            .4 makefile.
            .3 main.cpp \DTcomment{standard version of the program}.
            .3 main.
            .3 utilities.hpp.
            .3 makefile.
        }
        \caption{Directory tree for the watermarking project, for more details see the
        \href{https://github.com/germz01/PDS_project}{\textit{GitHub}} page.}
        \label{fig:directory_tree}
    \end{figure}
    % section simulating_the_stream_of_images (end)
% chapter implementation_structure_and_implementation_details (end)
